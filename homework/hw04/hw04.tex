\documentclass{article}
\usepackage{amsfonts,amssymb}
\usepackage[margin=1in]{geometry}

\title{CSCI 301, Math Exercises \# 4}
\author{YOUR NAME HERE}
\date{Due date:  Midnight, Nov 2}

\begin{document}

\maketitle

\begin{itemize}
  \item
Build deterministic finite automata
for each of the following languages.
\item
Create simple, meaningful automata
(rather than, {\em e.g.}, using the algorithm to create
a DFA from an NFA, which can be quite complex and confusing)
and explain how they work.
\item
In all
cases the alphabet is $\Sigma = \{0,1\}$.
\item
  Typeset your answers as neatly as possible with \LaTeX\ and
  Ti\textit{k}Z.
\item
  Turn in both your {\tt .tex} file and your {\tt .pdf} file.  Do not zip
  or combine them any other way into one file.
\end{itemize}

\begin{enumerate}

  \item   The language $\{100, 10, 011\}$. 

\item The set of 
  all strings that begin or end with a doubled digit, either 11 or 00.
  
\item
The set of all strings that have exactly one doubled digit in them.
  In other words, either 11 or 00 occurs in the string, but not both,
  and it only occurs once.

\item The set of all strings such that every block of four consecutive
  digits has at least two 0's in it.

\item The set of all strings beginning with a 1 such
  that, interpreted as a binary representation of an integer, it has a
  remainder of 1 when divided by 3.  For example, the binary number
  $1010_b$ is decimal $10$.  When you divide 10 by 3 you get a
  remainder of 1, so $1010$ is in the language.  However, the binary
  number $1111_b$ is decimal $15$.  When you divide 15 by 3 you get a
  remainder of 0, so $1111$ is not in the language.

  Hint: if you have a binary string, such as $1100_b$, which is
  $12$ in decimal, what happens if you add a 0 to the right end?
  You get $11000_b$ which in decimal is $24$.  What happens if
  you add a 1 to the right end?  You get $11001_b$ which is decimal
  $25$.  Think carefully about all cases:  What happens to the remainder
  when you add a 0?  What happens when you add a 1?

\end{enumerate}


\end{document}
