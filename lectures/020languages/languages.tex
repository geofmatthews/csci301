\documentclass{beamer}
\usepackage{fancyvrb}
\usepackage{hyperref}
\usepackage{alltt}
\usepackage{graphicx}

\newtheorem{theo}{Theorem}[section]


\newcommand{\myfig}[1]{\centerline{\includegraphics[scale=0.25]{figures/#1.png}}}
\newcommand{\myfigt}[1]{\centerline{\includegraphics[scale=0.2]{figures/#1.png}}}

\newcommand{\nats}{\ensuremath{\mathbb{N}}}
\newcommand{\nni}{\ensuremath{\mathbb{N}^0}}
\newcommand{\trans}[5]{
\begin{tabular}{|c|c|c|c|c|}\hline
#1 & #2 & #3 & #4 & #5 \\\hline
\end{tabular}
}

\newcommand{\arr}{&\rightarrow&}
\newcommand{\darr}{&\Rightarrow&}
\newcommand{\ar}{\ensuremath{\rightarrow}}
\newcommand{\dar}{\ensuremath{\Rightarrow}}
\newcommand{\bee}{\begin{eqnarray*}}
\newcommand{\eee}{\end{eqnarray*}}
\newcommand{\emptystring}{\ensuremath{\epsilon}}

\newcommand{\bi}{\begin{itemize}}
\newcommand{\li}{\item}
\newcommand{\ei}{\end{itemize}}

\newcommand{\sect}[1]{
\section{#1}
\begin{frame}[fragile]\frametitle{#1}
}
\newcommand{\sectc}[1]{
\section{#1}
\begin{frame}[fragile]\frametitle{#1}
\begin{columns}
}
\newcommand{\nc}[1]{\column{#1\textwidth}}
\newcommand{\ec}{\end{columns}}

\mode<presentation>
{
%  \usetheme{Madrid}
  % or ...

%  \setbeamercovered{transparent}
  % or whatever (possibly just delete it)
}

\usepackage[english]{babel}

\usepackage[latin1]{inputenc}

\title[Languages]
{
Languages
}

\subtitle{} % (optional)

\author[Geoffrey Matthews]
{Geoffrey Matthews}
% - Use the \inst{?} command only if the authors have different
%   affiliation.

\institute[WWU/CS]
{
  Department of Computer Science\\
  Western Washington University
}
% - Use the \inst command only if there are several affiliations.
% - Keep it simple, no one is interested in your street address.

\date{\today}

% If you have a file called "university-logo-filename.xxx", where xxx
% is a graphic format that can be processed by latex or pdflatex,
% resp., then you can add a logo as follows:

%\pgfdeclareimage[height=0.5cm]{university-logo}{WWULogoProColor}
%\logo{\pgfuseimage{university-logo}}

% If you wish to uncover everything in a step-wise fashion, uncomment
% the following command: 

%\beamerdefaultoverlayspecification{<+->}

\begin{document}

\begin{frame}
  \titlepage
\end{frame}


\newcommand{\myref}[1]{\small\item\url{#1}}
\newcommand{\myreft}[1]{\footnotesize\item\url{#1}}

%\begin{frame}
%  \frametitle{Outline}
%  \tableofcontents
%  % You might wish to add the option [pausesections]
%\end{frame}

\sect{Readings}

\begin{itemize}

\myreft{http://cglab.ca/~michiel/TheoryOfComputation/TheoryOfComputation.pdf}

\myreft{http://en.wikipedia.org/wiki/Formal_language}
\myreft{https://en.wikipedia.org/wiki/Automata_theory}

\myreft{http://users.utu.fi/jkari/automata/fullnotes.pdf}
\myreft{https://www.tutorialspoint.com/automata_theory/index.htm}
\myreft{https://www.geeksforgeeks.org/theory-of-computation-automata-tutorials/}
\myreft{https://cs.stanford.edu/people/eroberts/courses/soco/projects/2004-05/automata-theory/basics.html}
\myreft{https://www.iitg.ernet.in/dgoswami/Flat-Notes.pdf}

\myreft{https://www.youtube.com/watch?v=58N2N7zJGrQ&list=PLBlnK6fEyqRgp46KUv4ZY69yXmpwKOIev}
\end{itemize}

\end{frame}

\sect{Strings}

\bi
\li A finite set $A$ of symbols is given as the {\bf alphabet}
\li A {\bf string} or {\bf word} or {\bf sentence} is a finite sequence of symbols from the alphabet.
\li The {\bf length} of a string is denoted $|s|$
\li $\emptystring$ denotes the empty string. $|\emptystring|=0$
\bigskip
\li $\emptystring$, $a$, $abbca$, and $bccb$ are strings over the alphabet $\{a,b,c\}$
\li $\emptystring$, $110101$ and $0011$ are strings over the alphabet $\{0,1\}$
\li $\emptystring$, ``the black cat'' and ``cat cat the the'' are strings over the alphabet $\{$black, cat, the$\}$
\ei

\end{frame}

\sect{Concatenation}
\bi
\li The concatenation of two strings is the string obtained by placing them next to each other.
\li The concatenation of $aaa$ and $bccb$ is $aaabccb$
\li The concatenation of $s$ with itself $n$ times is denoted $s^n$
\li $(ab)^2 = abab$, $(aba)^3 = abaabaaba$, $(ab)^0 = \emptystring$
\ei
\end{frame}

\sect{Languages}
\bi
\li A set of strings over an alphabet is a {\bf language}.
\li Languages over $\{a,b\}$:

$\emptyset$, $\{\emptystring\}$, $\{b\}$, $\{\emptystring, abb, aaaa\}$,
$\{a^n : n \in \nni\}= \{\emptystring, a, aa, aaa, aaaa, \ldots\}$,
$\{ab^n : n \in \nni\}= \{ a, ab, abb, abbb, \ldots\}$,
$\{(ab)^n : n \in \nni\}= \{\emptystring, ab, abab, ababab, \ldots\}$,
$\{a^{n^2} : n \in \nni\}=\{\emptystring, a, aaaa, aaaaaaaaa, ...\}$,
$\{a^nb^{2n} : n\in\nni\}=\{\emptystring, abb, aabbbb, aaabbbbbb, ...\}$

\li Since they are sets, we can make new languages from old with:

 \hfill $L\cup M$ \hfill
 $L\cap M$ \hfill
 $L - M$ \hfill
 $\overline{L}$ \hfill\mbox{~}

\ei
\end{frame}

\sect{Product of Languages}
\bi
\li The product of languages $L$ and $M$ is
\[
LM = \{st : s\in L \wedge t \in M\}
\]
\li If $L=\{ab, bb\}$ and $M = \{a,b,c\}$ then
$LM = \{aba, abb, abc, bba, bbb, bbc\}$
\li Is it always the case that $|LM|=|L||M|$?
\pause
\li If $L=\{a,ab\}$ and $M=\{\emptystring, b\}$, then
$LM = \{a,ab,abb\}$
\pause
\li If $L=\{a, ab\}$ and $M=\{a,ba\}$, then
$LM = \{aa,aba,abba\}$
\pause
\li Why is it always the case that $|L\times M| = |L|\times|M|$?

\ei

\end{frame}

\sect{Properties of Language Products}
\bi
\li $L\{\emptystring\} = \{\emptystring\}L =  L$
\li $L\emptyset = \emptyset L = \emptyset$
\ei


\end{frame}

\sect{Product of a language with itself}
\bi
\li $L^n = \{s_1s_2s_3\ldots s_k : k\in\nni \wedge \forall i, s_i\in L\}$
\li If $L = \{a, bb\}$, then

\li $L^0 = \{\emptystring\}$
\li $L^1 = L = \{a,bb\}$
\li $L^2 = LL = \{aa,abb,bba,bbbb\}$

\ei
\end{frame}

\sect{Closure of a Language (Kleene Star)}
\bi
\li The {\bf closure} $L^*$ of a language $L$ is 
\bee
L^* &=& \bigcup_{i=0}^{\infty} L^i\\
   &=& L^0 \cup L^1 \cup L^2 \cup \ldots
\eee
\li The {\bf positive closure} $L^+$ of a language $L$ is 
\bee
L^+ &=& \bigcup_{i=1}^{\infty} L^i\\
   &=& L^1 \cup L^2 \cup L^3 \cup \ldots
\eee

\ei


\end{frame}

\sect{Properties of Closure}
\bi
\li $\emptyset^* = \{\emptystring\}^* = \{\emptystring\}$
\li $\emptystring \in L$ if and only if $L^+ = L^*$
\li $L^* = L^*L^* = (L^*)^*$
\li $(L^*M^*)^* = (L\cup M)^*$
\li $L(ML)^* = (LM)^*L$

\ei

\end{frame}
\sect{String Substitution}

\bi
\li Start with the string $ABBA$

\li If we make the substitutions $A\ar a$ and $B\ar b$
\li $ABBA \dar abba$

\li If we make the substitutions $A\ar ab$ and $B\ar ba$
\li  $ABBA \dar abbabaab$

\li If we make the substitutions $A\ar bab$ and $B\ar bbb$
\li  $ABBA \dar babbbbbbbbab$

\ei

\end{frame}


\sect{Formal Grammars}
\bi
\li A set of {\bf terminals}, e.g. $\{$the,cat,sat,on,mat$\}$
\li A set of {\bf nonterminals}, or {\bf variables}, e.g. $\{S,N\}$
\li A special nonterminal, the {\bf start symbol}, e.g. $S$
\li A set of {\bf production rules}:
\begin{eqnarray*}
S \arr \mbox{the~} N \mbox{~sat on the~} N\\
N\arr \mbox{cat}\\
N\arr \mbox{mat}
\end{eqnarray*}
\li A {\bf derivation} is any string we get by starting with the start
symbol and repeatedly making a single substitution until we only have
terminals.
\li $S\dar \mbox{~the~} N \mbox{~sat on the~} N \dar \mbox{~the cat sat
  on the~} N \dar \mbox{~the cat sat on the mat}$
\li $S\dar \mbox{~the~} N \mbox{~sat on the~} N \dar \mbox{~the mat sat
  on the~} N \dar \mbox{~the mat sat on the mat}$
\ei
\end{frame}

\sect{Vertical bar means ``or''}
This grammar:
\begin{eqnarray*}
S \arr \mbox{the~} N \mbox{~sat on the~} N\\
N\arr \mbox{cat}\\
N\arr \mbox{mat}
\end{eqnarray*}
is equivalent to this grammar:
\begin{eqnarray*}
S \arr \mbox{the~} N \mbox{~sat on the~} N\\
N\arr \mbox{cat}\ |\  \mbox{mat}
\end{eqnarray*}
\end{frame}

\sect{Rules can be recursive}

\begin{eqnarray*}
S \arr S \mbox{~and~} S\\
S \arr \mbox{the~} N \mbox{~sat on the~} N\\
N\arr \mbox{cat}\ |\  \mbox{mat}
\end{eqnarray*}

\end{frame}

\end{document}
