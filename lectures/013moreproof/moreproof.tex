\documentclass{beamer}
\usetheme{Singapore}

\usepackage{amsmath,amssymb,latexsym}
\usepackage{graphicx}
\usepackage{fancyvrb}
\usepackage{hyperref}

\newcommand{\bi}{\begin{itemize}}
\newcommand{\ii}{\item}
\newcommand{\ei}{\end{itemize}}
\newcommand{\Show}[1]{\psshadowbox{#1}}

\newcommand{\set}[1]{\ensuremath{\left\{ #1 \right\}}}
\newcommand{\nats}{\ensuremath{\mathbb{N}}}
\newcommand{\nni}{\ensuremath{\mathbb{N}^0}}
\newcommand{\ints}{\ensuremath{\mathbb{Z}}}
\newcommand{\power}{\ensuremath{\mathcal{P}}}
\renewcommand{\neg}{\sim}
\newcommand{\xor}{\oplus}
\newcommand{\then}{\ensuremath{\Rightarrow}}
\newcommand{\lcm}{\mbox{lcm}}
\newcommand{\QED}{\mbox{}\hfill\ensuremath{\blacksquare}}

\newcommand{\grf}[2]{\centerline{\includegraphics[width=#1\textwidth]{#2}}}
\newcommand{\tw}{\textwidth}
\newcommand{\bc}{\begin{columns}}
\newcommand{\ec}{\end{columns}}
\newcommand{\cc}[1]{\column{#1\textwidth}}

\newcommand{\bfr}[1]{\begin{frame}[fragile]\frametitle{{ #1 }}}
\newcommand{\efr}{\end{frame}}

\newcommand{\cola}[1]{\begin{columns}\begin{column}{#1\textwidth}}
\newcommand{\colb}[1]{\end{column}\begin{column}{#1\textwidth}}
\newcommand{\colc}{\end{column}\end{columns}}

\title{Book of Proof: Part III, More on Proof}

\RecustomVerbatimEnvironment{Verbatim}{Verbatim}{frame=single}

\begin{document}
\begin{frame}
\maketitle

\end{frame}

\bfr{If-and-Only-If Proof}

\begin{center}

  {\bf Outline for If-and-Only-If Proof}
  \fbox{\parbox{0.8\textwidth}{
      {\bf Proposition} $P$ if and only if $Q$.

      {\it Proof.}
\\\\
      ``Only if''
      
      [Prove $P\then Q$ by whatever means you can.]
\\\\
      ``If''
      
      [Prove $Q\then P$ by whatever means you can.]
    }
  }
\end{center}

\end{frame}

\bfr{Equivalent Statements}

{\bf Theorem} Suppose $A$ is an $n\times n$ matrix.  The following
statements are equivalent:
\renewcommand{\theenumi}{\alph{enumi}}
\begin{enumerate}
\item $A$ is invertible.
\item $Ax=b$ has a unique solution for every $b\in\mathbb{R}^n$.
\item $Ax=0$ has only the trivial solution.
\item The reduced row echelong form of $A$ is $I_n$.
\item $\det(A)\neq 0$.
\item The matrix $A$ does not have 0 as an eigenvector.
\end{enumerate}

\end{frame}

\bfr{Equivalent Statements}
\[
\begin{array}{ccccc}
  a & \Rightarrow & b & \Rightarrow & c\\
  \Uparrow &&&&\Downarrow \\
  f & \Leftarrow & e & \Leftarrow &  d
\end{array}
\]
\pause\vfill
\[
\begin{array}{ccccc}
  a & \Rightarrow & b & \Leftrightarrow & c\\
  \Uparrow &&\Downarrow&&\\
  f & \Leftarrow & e & \Leftrightarrow &  d
\end{array}
\]
\pause\vfill
\[
\begin{array}{ccccc}
  a & \Leftrightarrow & b & \Leftrightarrow & c\\
   &&\Updownarrow  \\
  f & \Leftrightarrow & e & \Leftrightarrow &  d
\end{array}
\]


\end{frame}

\bfr{Existence Proofs}

{\bf Proposition} There exists an even prime number.
\pause

{\it Proof.} Two is an even prime number.\QED

\pause\vfill

{\bf Proposition}  There exists an integer that can be expressed as
the sum of two perfect cubes in two different ways.
\pause

{\it Proof.}
\begin{align*}
  1^3 + 12^3 &= 1729\\
  9^3 + 10^3 &= 1729
\end{align*}\QED
\end{frame}


\bfr{Example}

{\bf Proposition 7.1} If $a,b\in\nats$ then \\there exist
$k,\ell\in\ints$ for which $\gcd(a,b)=ak+b\ell$.

\vfill

For example:
\begin{description}

  \item $\gcd(12,18) = 6$ and $6 = (-1)12 + (1)18$
  \item $\gcd(9,21) = 3$ and $3= (-2)9 + (1)21$

\end{description}



\end{frame}


\bfr{Example}

{\bf Proposition 7.1} If $a,b\in\nats$ then \\there exist
$k,\ell\in\ints$ for which $\gcd(a,b)=ak+b\ell$.

\vfill

{\it Proof.}  Suppose $a,b\in\nats$.

Consider the set $A=\set{ax+by : x,y\in\ints}$.

$A$ contains positive integers and 0.

Let $d\in A$ be the smallest positive integer.

$d = ak + b\ell$ for some $k,\ell\in\ints$.

We will show that $d=\gcd(a,b)$.

First, prove that $d\mid a$ and $d\mid b$.

Then show that it is the largest such number.


\end{frame}

\bfr{Example}

{\bf Proposition 7.1} If $a,b\in\nats$ then \\there exist
$k,\ell\in\ints$ for which $\gcd(a,b)=ak+b\ell$.

\vfill
    {\it Proof (continued).}

    $d=ak+b\ell$ is the smallest positive element of
$A=\set{ax+by : x,y\in\ints}$.

Show that $d \mid a$.

Use division algorithm: $a = qd + r$.
\begin{align*}
  r &= a - qd\\
  &= a - q(ak+b\ell)\\
  &= a(1-qk) + b(-q\ell)
\end{align*}
So $r\in A$, $0\leq r< d$, so $r=0$.

So $a = qd + r = qd$ and so $d\mid a$.
\pause

A similar argument shows $d\mid b$.
\end{frame}

\bfr{Example}

{\bf Proposition 7.1} If $a,b\in\nats$ then \\there exist
$k,\ell\in\ints$ for which $\gcd(a,b)=ak+b\ell$.

\vfill
{\it Proof (continued).}  $d=ak+b\ell$ is the smallest positive element of
$A=\set{ax+by : x,y\in\ints}$, and $d \mid a$ and $d \mid b$.

\begin{align*}
a &=   \gcd(a,b) \cdot m\\
b &= \gcd(a,b)\cdot n\\
d &= ak + b\ell \\
&= \gcd(a,b)\cdot mk + \gcd(a,b)\cdot n\ell\\
&= \gcd(a,b)(mk+n\ell)\\
d &\geq \gcd(a,b)\\
d &= \gcd(a,b)  
\end{align*}
\QED

\end{frame}

\bfr{Proofs involving sets}

  {\bf How to show $a\in\set{x:P(x)}$}
  
  \fbox{\parbox{0.6\textwidth}{
    Show that $P(a)$ is true.\QED
  }}

  \vfill
  
  {\bf How to show $a\in\set{x\in S:P(x)}$}
  
  \fbox{\parbox{0.6\textwidth}{
    1.    Verify that $a\in S$.\\    
    2.    Show that $P(a)$ is true.\QED
  }}

\end{frame}

\bfr{Proofs involving sets}

{\bf How to Prove $A\subseteq B$}

{\bf (Direct approach)}
  
  \fbox{\parbox{0.6\textwidth}{
      {\it Proof.}  Suppose $a\in A$.

      $\vdots$

      Therefore $a\in B$. \QED
  }}

  \vfill

  {\bf How to Prove $A\subseteq B$}


{\bf (Contrapositive approach)}
  
  \fbox{\parbox{0.6\textwidth}{
      {\it Proof.}  Suppose $a\not \in B$.

      $\vdots$

      Therefore $a\not \in A$. \QED
  }}



\end{frame}

\bfr{Proofs involving sets}

{\bf How to Prove $A = B$}

  
  \fbox{\parbox{0.6\textwidth}{
      {\it Proof.}

      [Prove that $A\subseteq B$.]

      [Prove that $B\subseteq A$.]

 \QED
  }}
\end{frame}

\bfr{Disproof}

\fbox{\parbox{0.9\textwidth}{
    {\bf How to disprove $P$:}

    Prove $\neg P$. \QED
}}

\pause\vfill

\fbox{\parbox{0.9\textwidth}{
    {\bf How to disprove $\forall x\in S, P(x)$:}

    Produce an example of $x\in S$ where $P(x)$ is false. \QED
}}


\pause\vfill

\fbox{\parbox{0.9\textwidth}{
    {\bf How to disprove $P(x)\then Q(x)$:}

    Produce an example of $x$ where $P(x)$ is true but $Q(x)$ is false.
    \QED
}}


\end{frame}

\bfr{Proving facts about \nats}

\grf{1.0}{sumofodds.png}

\end{frame}

\bfr{Proving facts about \nats}

For all $n\in\nats$,

\begin{align*}
  1 + 3 + 5 + 7 + ... + (2n-1) &=n^2  \\\\
  \sum_{i=1}^n (2i-1)  &= n^2
\end{align*}

\begin{itemize}
  \item
Does not appear to be a conditional we can work from.
\item
Negating it does not lead to an easy contradiction.
\end{itemize}

\end{frame}

\bfr{Mathematical Induction}

\grf{0.75}{domino.png}

\end{frame}

\bfr{Mathematical Induction}


  {\bf Outline for Proof by Induction}

  \fbox{\parbox{\textwidth}{
      {\bf Proposition}  The statements $S_1,S_2,S_3,\ldots$ are all true.

      {\it Proof.}

      (1) Prove that $S_1$ is true.

      (2) Prove that for $k\in\nats$, $S_k \Rightarrow S_{k+1}$ is true.
      \QED
  }}
  \vfill\pause
  
  \fbox{\parbox{\textwidth}{
      {\bf Proposition}  For all $n\in\nats$, $S_n$.

      {\it Proof.}

      (1) Prove that $S_1$ is true.

      (2) Prove that for $k\in\nats$, $S_k \Rightarrow S_{k+1}$ is true.
      \QED
  }}

\end{frame}

\bfr{Example Proof by Induction}

{\bf Proposition} If $n\in\nats$, then $1+3+5+7 +...+(2n-1) = n^2$.

(1) If $n=1$, then we need to prove $1=1^2$, which is obviously true.

(2) Assume
\begin{align*}
1+3+5+7 +...+(2k-1) &= k^2 &\mbox{for some $k\in\nats$.}
\end{align*}


\vfill
\centerline{$\vdots$}
\vfill

Therefore,
\[
1+3+5+7 +...+(2(k+1)-1) = (k+1)^2
\]

\QED

\end{frame}

\bfr{Example Proof by Induction}

{\bf Proposition} If $n\in\nats$, then $1+3+5+7 +...+(2n-1) = n^2$.

(1) If $n=1$, then $1=1^2$, which is true.

(2) Assume $1+3+5+7 +...+(2k-1) = k^2$ for some $k\in\nats$.

Then
\begin{align*}
  1+3+5+7 +... + 2(k+1) -1 &=\\
{\color{blue}  1+3+5+7 +...+(2k-1)} + (2(k+1) -1) &= {\color{blue} k^2} + (2(k+1) -1)\\
  &= k^2 + 2d + 1\\
  &= (k+1)^2
\end{align*}

Therefore,
\[
1+3+5+7 +...+(2(k+1)-1) = (k+1)^2
\]
\QED

\end{frame}

\bfr{Example Proof by Induction}

{\bf Proposition} For all $n\in\nats$,  \[\sum_{i=1}^n (2i-1) = n^2\]

(1) If $n=1$, then we need to prove $1=1^2$, which is obviously true.

(2) Assume, for some $k\in\nats$ (the induction hypothesis):
\begin{align*}
\sum_{i=1}^k (2i-1) &= k^2 
\end{align*}


\vfill
\centerline{$\vdots$}
\vfill

Therefore,
\begin{align*}
\sum_{i=1}^{k+1} (2i-1) & = (k+1)^2
\end{align*}

\QED

\end{frame}

\bfr{Example Proof by Induction}

{\bf Proposition} If $n\in\nats$, then \[\sum_{i=1}^n (2i-1) = n^2\]

(1) If $n=1$, then we need to prove $1=1^2$, which is obviously true.

(2)Assume
\begin{align*}
\sum_{i=1}^k (2i-1) &= k^2 &\mbox{for some $k\in\nats$.}\\
  \sum_{i=1}^{k+1} (2i-1) &= 
  {\color{blue}\sum_{i=1}^k(2i-1)} + (2(k+1) -1)\\
  &= {\color{blue}k^2} + 2k + 1 &\mbox{by induction hypothesis}\\
 &= (k+1)^2
\end{align*}
\QED

\end{frame}



\bfr{Example Proof by Induction}


{\bf Proposition} If $n\in\nni$, then $5\mid (n^5-n)$.

{\sl Proof.}

(1) If $n=0$, then we need to prove $5\mid (0^5-0)$, which is true.

(2) Assume $5\mid (k^5-k)$ for some $k\in\nni$.

\vfill

$\vdots$
\vfill

Therefore $5\mid((k+1)^5 - (k+1))$.
\QED

\vfill

{\sl What can we get from definitions?}

\end{frame}

\bfr{Example Proof by Induction}

{\bf Proposition} If $n\in\nni$, then $5\mid (n^5-n)$.

{\sl Proof.}

(1) If $n=0$, then we need to prove $5\mid (0^5-0)$, which is true.

(2) Assume $5\mid (k^5-k)$ for some $k\in\nni$.

Then $(k^5-k) = 5a$ for some $a\in\nats$.
\vfill

$\vdots$
\vfill
Then $((k+1)^5 - (k+1)) = 5b$ for some  $b\in\nats$.

Therefore $5\mid((k+1)^5 - (k+1))$.
\QED

\end{frame}

\bfr{Example Proof by Induction}

{\bf Proposition} If $n\in\nni$, then $5\mid (n^5-n)$.

{\sl Proof.}

(1) If $n=0$, then we need to prove $5\mid (0^5-0)$, which is true.

(2) Assume $5\mid (k^5-k)$ for some $k\in\nni$.

Then $(k^5-k) = 5a$ for some $a\in\nats$.
\begin{align*}
  (k+1)^5 - (k+1) &= k^5 + 5k^4 + 10k^3 + 10k^2 + 5k + 1 - k - 1\\
  &= {\color{blue}(k^5-k)} + 5k^4 + 10k^3 + 10k^2 + 5k\\
  &= {\color{blue}5a} + 5k^4 + 10k^3 + 10k^2 + 5k\\
  &= 5(a + k^4 + 2k^3 + 2k^2 + k)
\end{align*}
Then $((k+1)^5 - (k+1)) = 5b$ for some  $b\in\nats$.

Therefore $5\mid((k+1)^5 - (k+1))$.
\QED
\end{frame}

\bfr{SymPy}

For some help with large algebraic expressions:
\url{http://www.sympy.org/}

\vfill
\begin{Verbatim}[frame=single]
>>> from sympy import *
>>> k = symbols('k')
>>> expr = (k+1)^5 - (k + 1)
>>> expand(expr)
k**5 + 5*k**4 + 10*k**3 + 10*k**2 + 4*k
>>> expand(expr - (k**5 - k))
5*k**4 + 10*k**3 + 10*k**2 + 5*k
\end{Verbatim}
  

\end{frame}
  
\bfr{Strong Induction}



  {\bf Outline for Proof by Induction}

  \fbox{\parbox{\textwidth}{
      {\bf Proposition}  The statements $S_1,S_2,S_3,\ldots$ are all true.

      {\it Proof.}

      (1) Prove that $S_1$ is true.

      (2) Prove that for $k\in\nats$, $S_k \Rightarrow S_{k+1}$ is true.
      \QED
  }}
  \vfill

  {\bf Outline for Proof by Strong Induction}

  \fbox{\parbox{\textwidth}{
      {\bf Proposition}  The statements $S_1,S_2,S_3,\ldots$ are all true.

      {\it Proof.}

      (1) Prove that $S_1$ is true.  (Or the first several $S_n$.)

      (2) Prove that for $k\in\nats$, $(S_1\land S_2 \land S_3 \land ... \land S_k) \Rightarrow S_{k+1}$ is true.
      \QED
  }}

\end{frame}

\bfr{Smallest Counterexample}

  {\bf Outline for Proof by Induction}

  \fbox{\parbox{\textwidth}{
      {\bf Proposition}  The statements $S_1,S_2,S_3,\ldots$ are all true.

      {\it Proof.}

      (1) Prove that $S_1$ is true.

      (2) Prove that for $k\in\nats$, $S_k \Rightarrow S_{k+1}$ is true.
      \QED
  }}
  \vfill

  {\bf Outline for Proof by Smallest Counterexample}

  \fbox{\parbox{\textwidth}{
      {\bf Proposition}  The statements $S_1,S_2,S_3,\ldots$ are all true.

      {\it Proof.}

      (1) Prove that $S_1$ is true. 

      (2) Suppose that not every $S_n$ is true.

      (3) Let $S_k$ be the smallest false one.

      (4) Then $S_{k-1}$ is true and $S_k$ is false.

      (5) Use this to get a contradiction.
      \QED
  }}

\end{frame}

\end{document}
