\documentclass{beamer}
\usepackage{fancyvrb}
\usepackage{hyperref}
\usepackage{alltt}
\usepackage{graphicx}
\usepackage{tikz}
\usepackage{tikz-qtree}

\newcommand{\sect}[1]{
\section{#1}
\begin{frame}[fragile]\frametitle{#1}
}

\newtheorem{theo}{Theorem}[section]

\newcommand{\myfig}[1]{\centerline{\includegraphics[scale=0.25]{figures/#1.png}}}
\newcommand{\myfigt}[1]{\centerline{\includegraphics[scale=0.2]{figures/#1.png}}}
\newcommand{\myfigg}[1]{\centerline{\includegraphics[scale=0.125]{figures/#1.png}}}

\newcommand{\trans}[5]{
\begin{tabular}{|c|c|c|c|c|}\hline
#1 & #2 & #3 & #4 & #5 \\\hline
\end{tabular}
}

\newcommand{\arr}{&\rightarrow&}
\newcommand{\darr}{&\Rightarrow&}
\newcommand{\ar}{\ensuremath{\rightarrow}}
\newcommand{\dar}{\ensuremath{\Rightarrow}}
\newcommand{\bee}{\begin{eqnarray*}}
\newcommand{\eee}{\end{eqnarray*}}
\newcommand{\mt}{\ensuremath{\epsilon}}

\newcommand{\deriv}[1]{\ensuremath{\stackrel{#1}{\Longrightarrow}}}
\newcommand{\bi}{\begin{itemize}}
\newcommand{\li}{\item}
\newcommand{\ei}{\end{itemize}}


\newcommand{\lrtable}{
\begin{tabular}{lrlr}
Stack & Input & Rule & Peek\\\hline
 $\$ aCd$ & $\$$&S\ar aCd & \\
 $\$ ac^+C$ & $d \$$ & C \ar cC & \\
 $\$ ac^+$&$d \$$ & C \ar c & d\\
 $\$ bCD$&$ \$$ & S\ar bCD & \\
 $\$ bCd$& $ \$$& D\ar d & \\
 $\$ bc^+C$&$d \$$ & C \ar cC & \\
 $\$ bc^+$&$d \$$& C \ar c& d\\\hline
\end{tabular}
}


\mode<presentation>
{
%  \usetheme{Madrid}
  % or ...

%  \setbeamercovered{transparent}
  % or whatever (possibly just delete it)
}

\usepackage[english]{babel}

\usepackage[latin1]{inputenc}

\title[Notes on LR Parsing]
{
 Notes on LR  Parsing
}

\subtitle{} % (optional)

\author[Geoffrey Matthews]
{Geoffrey Matthews}
% - Use the \inst{?} command only if the authors have different
%   affiliation.

\institute[WWU/CS]
{
  Department of Computer Science\\
  Western Washington University
}
% - Use the \inst command only if there are several affiliations.
% - Keep it simple, no one is interested in your street address.

\date{\today}

% If you have a file called "university-logo-filename.xxx", where xxx
% is a graphic format that can be processed by latex or pdflatex,
% resp., then you can add a logo as follows:

%\pgfdeclareimage[height=0.5cm]{university-logo}{WWULogoProColor}
%\logo{\pgfuseimage{university-logo}}

% If you wish to uncover everything in a step-wise fashion, uncomment
% the following command: 

%\beamerdefaultoverlayspecification{<+->}

\begin{document}

\begin{frame}
  \titlepage
\end{frame}


\newcommand{\myref}[1]{\small\item\url{#1}}
\newcommand{\myreft}[1]{\footnotesize\item\url{#1}}

%\begin{frame}
%  \frametitle{Outline}
%  \tableofcontents
%  % You might wish to add the option [pausesections]
%\end{frame}

\sect{Readings}

\begin{itemize}

\myreft{http://www.cs.rochester.edu/~nelson/courses/csc_173/grammars/cfg.html}

\myreft{http://en.wikipedia.org/wiki/Context-free_grammar}

\myreft{http://en.wikipedia.org/wiki/Context-free_language}
\myreft{http://en.wikipedia.org/wiki/Parsing}

\myreft{http://en.wikipedia.org/wiki/Pushdown_automata}
\myreft{http://en.wikipedia.org/wiki/LR_parser}
\myreft{https://parasol.tamu.edu/~rwerger/Courses/434/lec12-sum.pdf}
\end{itemize}

\end{frame}

\sect{Bottom up parsing of CFGs}
\begin{columns}
\column{0.6\textwidth}
\bi
\li We start with the input and attempt to build the parse tree.
\li If we begin with the input and attempt to build the tree 
above it, we are doing {\bf bottom-up} parsing.
\li Equivalently, we try to constuct a rightmost derivation from right
to left, scanning the input left to right.
\ei
\column{0.4\textwidth}
\bee
S \arr SA\ | \ \mt\\
A \arr AA\ | \ a
\eee
\myfig{derivationtreelr}
\end{columns}
\vfill
\[
S \deriv{S\ar SA} SA \deriv{A\ar AA} SAA \deriv{A\ar a} SAa \deriv{A\ar a} Saa \deriv{S\ar\epsilon} aa
\]
\end{frame}


\sect{$LR(k)$ grammars}
\bi
\li $LR(k)$ means we find a rightmost derivation by scanning the input
left to right, and have to lookahead at most $k$ symbols.
\ei
\end{frame}

\sect{$LR$ parsing:  Shift and Reduce}

\begin{description}
\li[Shift:] move character from input to stack
\li[Reduce:] if stack holds RHS of a rule, replace with LHS
\end{description}

\begin{columns}
\column{0.5\textwidth}
\bee
S \arr AB\\
A \arr a\\
B \arr b
\eee
\myfig{simplelrtree}

\column{0.5\textwidth}

\begin{tabular}{|lr|l|}\hline
Stack & Input & Rule \\\hline
\$ & ab\$ & shift\\
\$a & b\$ & A\ar a\\
\$A & b\$ & shift\\
\$Ab & \$ & B\ar b\\
\$AB & \$ & S\ar AB\\
\$S & \$ & accept\\\hline
\end{tabular}

\bigskip

$S\deriv{S\ar AB} AB \deriv{B\ar b} Ab \deriv{A\ar a} ab$
\end{columns}
\bigskip

\bi\li Note: At all times, stack+input=derivation string
\ei

\end{frame}

\sect{$LR$ parsing:  Shift and Reduce}

\begin{columns}
\column{0.5\textwidth}
\bee
S \arr SA\ | \ \mt\\
A \arr AA\ | \ a
\eee
\myfig{derivationtreelr}

\column{0.5\textwidth}
\begin{tabular}{|lr|l|}\hline
Stack & Input & Rule \\\hline
\$ & aa\$ & $S\ar\mt$\\
\$S & aa\$ & shift\\
\$Sa & a\$ & $A\ar a$\\
\$SA & a\$ & shift\\
\$SAa & \$ & $A\ar a$\\
\$SAA & \$ & $A\ar AA$\\
\$SA & \$ & $S\ar SA$\\
\$S & \$ & accept\\\hline
\end{tabular}

\end{columns}

\bigskip

$
S \deriv{S\ar SA} SA
 \deriv{A\ar AA} SAA 
\deriv{A\ar a} SAa 
\deriv{A\ar a} Saa
 \deriv{S\ar \epsilon} aa
$

\end{frame}

\sect{Another $LR$ parse}

\begin{columns}
\column{0.5\textwidth}
\bee
S \arr aSB \ | \ d\\
B \arr b
\eee


\myfig{lrtree}

\column{0.5\textwidth}
\begin{tabular}{|lr|l|}\hline
Stack & Input & Rule \\\hline
\$ & aadbb\$ & shift\\
\$a & adbb\$ & shift\\
\$aa & dbb\$ & shift\\
\$aad & bb\$ &  $S\ar d$\\
\$aaS & bb\$ &  shift\\
\$aaSb & b\$ &  $B\ar b$\\
\$aaSB & b\$ &  $S\ar aSB$\\
\$aS & b\$ &  shift\\
\$aSb & \$ &  $B\ar b$\\
\$aSB & \$ &    $S\ar aSB$\\
\$S & \$ &    accept\\\hline
\end{tabular}

\end{columns}

\bigskip

$
S \deriv{S\ar aSB} aSB
\deriv{B\ar b} aSb
\deriv{S\ar aSB} aaSBb
\deriv{B\ar b} aaSbb
\deriv{S\ar d} aadbb
$

\end{frame}

\sect{LR parsing arithmetic}
\footnotesize
\begin{columns}
\column{0.2\textwidth}
\bee
E \arr E + T \ | \ T\\
T \arr T * F \ | \ F\\
F \arr (E) \ | \ a\\
\\
E \darr 
E+T \\\darr 
E+T*F \\\darr 
E+T*a \\\darr 
E+F*a \\\darr 
E+a*a \\\darr 
T+a*a \\\darr 
F+a*a \\\darr 
a+a*a
\eee

\column{0.4\textwidth}
\begin{tabular}{|lr|l|}\hline
Stack & Input & Rule \\\hline
\$ & a+a*a\$ & shift\\
\$a & +a*a\$ & F\ar a\\
\$F & +a*a\$ & T\ar F\\
\$T & +a*a\$ & E\ar T\\
\$E & +a*a\$ & shift\\
\$E+ & a*a\$ & shift\\
\$E+a & *a\$ & F\ar a\\
\$E+F & *a\$ & T\ar F\\
\$E+T & *a\$ & shift\\
\$E+T* & a\$ & shift\\
\$E+T*a & \$ & F\ar a\\
\$E+T*F & \$ & T\ar T*F\\
\$E+T & \$ & E\ar E+T\\
\$E & \$ &    accept\\\hline
\end{tabular}
\column{0.33\textwidth}
\myfigt{arithmetictree}

\end{columns}

\end{frame}


\sect{LR(1) parsing}
\bi
\li The trick is to know when to shift and when to reduce.
\li Hopefully by looking at {\bf only one} symbol of the input.
\li Everything on the stack has already been examined.
\li We can use the entire stack to determine actions.
\li We do this by using a DFA to keep track of stack state.
\li We note each time a RHS appears on top of the stack.
\li If a RHS is on top of the stack, a reduction is {\em possible}.
\bi
\li We can then choose whether to shift or reduce.
\li Otherwise you must shift.
\ei

\ei

\end{frame}

\sect{$LR(1)$ parsing}
\begin{columns}
\column{0.5\textwidth}
\bee
S \arr AB\\
A \arr a\\
B \arr b
\eee
\myfig{simplelrtree}

\bigskip

\centerline{$S\deriv{S\ar AB} AB \deriv{B\ar b} Ab \deriv{A\ar a} ab$}

\column{0.5\textwidth}

\begin{tabular}{|lr|l|}\hline
Stack & Input & Rule \\\hline
\$ & ab\$ & shift\\
\$a & b\$ & A\ar a\\
\$A & b\$ & shift\\
\$Ab & \$ & B\ar b\\
\$AB & \$ & S\ar AB\\
\$S & \$ & accept\\\hline
\end{tabular}

\bigskip
\myfig{simplelrfsa}
\end{columns}
\bigskip

We will store the state of the DFA on the stack, too.

\end{frame}

\sect{$LR(1)$ parsing}
\begin{columns}
\column{0.25\textwidth}
\bee
S \arr AB\\
A \arr a\\
B \arr b
\eee
\column{0.25\textwidth}
\myfig{simplelrtree}

\column{0.5\textwidth}

\begin{tabular}{|lr|l|}\hline
Stack & Input & Rule \\\hline
0 & ab\$ & shift\\
0 a 1 & b\$ & A\ar a\\
0 A 2 & b\$ & shift\\
0 A 2 b 3 & \$ & B\ar b\\
0 A 2 B 4 & \$ & S\ar AB\\
0 S 5 & \$ & accept\\\hline
\end{tabular}
\end{columns}
\begin{columns}
\column{0.5\textwidth}

{$S\deriv{S\ar AB} AB \deriv{B\ar b} Ab \deriv{A\ar a} ab$}

\bigskip

\footnotesize
\begin{tabular}{|c|c|c|c|c|c|c|}\hline
  & a & b & A & B & S & \$ \\\hline
0 & 1 &   & 2 &&5&  \\\hline
1 &   & $A\ar a$& & &&\\\hline
2 &   & 3 &&4&&\\\hline
3 &   &   & & &&$B\ar b$ \\\hline
4 &   &   & &&&$S\ar AB$ \\\hline
5 &   &   & &&&accept \\\hline
\end{tabular}
\column{0.5\textwidth}

\myfig{simplelrfsa}
\end{columns}

\end{frame}


\sect{Left recursion: $ S \rightarrow Sa\ |\ a$}

\begin{tabular}{lr}
Stack & Input \\
0 & a a a a \$ \\
0 a 1 & a a a \$ \\
0 S 2 & a a a \$ \\
0 S 2 a 3 & a a  \$ \\
0 S 2 & a a  \$ \\
0 S 2 a 3 & a  \$ \\
0 S 2 & a  \$ \\
0 S 2 a 3 &  \$ \\
0 S 2 & \$ \\
\end{tabular}
\hfill
\begin{tabular}{|c|c|c|c|}\hline
 & $a$ & \$ & $S$ \\\hline
0 & 1 & & 2\\\hline
1 & $S\rightarrow a$ &&\\\hline
2 & 3 & accept& \\\hline
3 & $S \rightarrow Sa$ & $S \rightarrow Sa$ &\\\hline
\end{tabular}

\hfill
\myfig{lrparseexamples01}
\end{frame}


\sect{Right recursion:
$ S \rightarrow aS\ |\ a$}

\begin{tabular}{lr}
Stack & Input \\
0 & a a a a \$ \\
0 a 1 & a a a  \$ \\
0 a 1 a 1 & a a  \$ \\
0 a 1 a 1 a 1 & a  \$ \\
0 a 1 a 1 a 1 a 1 & \$ \\
0 a 1 a 1 a 1 S 2 & \$ \\
0 a 1 a 1 S 2 & \$ \\
0 a 1 S 2 & \$ \\
0 S 3 & \$ \\
\end{tabular}
\hfill
\begin{tabular}{|c|c|c|c|}\hline
 & $a$ & \$ & $S$ \\\hline
0 & 1 & & 3 \\\hline
1 & 1 & $S \rightarrow a$ & 2 \\\hline
2 & & $S\rightarrow aS$ & \\\hline
3 &  & accept & \\\hline
\end{tabular}

\hfill
\myfig{lrparseexamples02}
\end{frame}


\sect{Middle recursion: $S \rightarrow aSa\ |\ bSb\ |\ c$}
{\footnotesize
\begin{minipage}{0.45\textwidth}
\begin{tabular}{lr}
Stack & Input \\
0 & a b c b a \$ \\
0 a 1 & b c b a  \$ \\
0 a 1 b 2 & c b a  \$ \\
0 a 1 b 2 c 3 & b a  \$ \\
0 a 1 b 2 S 5 & b a  \$ \\
0 a 1 b 2 S 5 b 7 & a  \$ \\
0 a 1 S 4 & a  \$ \\
0 a 1 S 4 a 6 & \$ \\
0 S 8 & \$\\
\end{tabular}
\end{minipage}\begin{minipage}{0.5\textwidth}
\begin{tabular}{|c|c|c|c|c|c|}\hline
 & a & b & c & \$ & S \\\hline
0 & 1 & 2 & 9 && 8\\\hline
1 & 1 & 2 & 3 && 4\\\hline
2 & 1 & 2 & 3 && 5\\\hline
3 & $S\rightarrow c$& $S\rightarrow c$&&&\\\hline
4 & 6 &&&&\\\hline
5 &&7&&&\\\hline
6 & $S\rightarrow aSa$ & $S\rightarrow aSa$ &&$S\rightarrow aSa$ &\\\hline
7 & $S\rightarrow bSb$ & $S\rightarrow bSb$ &&$S\rightarrow bSb$&\\\hline
8 &&&&accept&\\\hline
9 &&&&$S\rightarrow c$&\\\hline
\end{tabular}
\end{minipage}
}

\includegraphics[scale=0.15]{figures/lrparseexamples03.png}
\end{frame}

\sect{LR(1) parsing, a more complex example}

\begin{columns}
\column{0.5\textwidth}
\bee
S \arr aCd \ | \ bCD\\
C \arr cC \ | \ c\\
D \arr d
\eee


\myfig{accdtree}

\column{0.5\textwidth}
\begin{tabular}{|lr|l|}\hline
Stack & Input & Rule \\\hline
\$ & acccd\$ & shift\\
\$a & cccd\$ & shift\\
\$ac & ccd\$ & shift\\
\$acc & cd\$ & shift\\
\$accc & d\$ & C\ar c\\
\$accC & d\$ & C\ar cC\\
\$acC & d\$ & C\ar cC\\
\$aC & d\$ & shift\\
\$aCd & \$ & S \ar aCd\\
\$S & \$ &    accept\\\hline
\end{tabular}

\end{columns}

\bigskip

$S \dar aCd \dar acCd \dar accCd \dar acccd$

\end{frame}

\sect{LR(1) parsing}

\begin{columns}
\column{0.5\textwidth}
\bee
S \arr aCd \ | \ bCD\\
C \arr cC \ | \ c\\
D \arr d
\eee


\myfig{bccdtree}

\column{0.5\textwidth}
\begin{tabular}{|lr|l|}\hline
Stack & Input & Rule \\\hline
\$ & bcccd\$ & shift\\
\$b & cccd\$ & shift\\
\$bc & ccd\$ & shift\\
\$bcc & cd\$ & shift\\
\$bccc & d\$ & C \ar c\\
\$bccC & d\$ & C \ar cC\\
\$bcC & d\$ & C \ar cC\\
\$bC & d\$ & shift\\
\$bCd & \$ & D\ar d\\
\$bCD & \$ & S\ar bCD\\
\$S & \$ &    accept\\\hline
\end{tabular}

\end{columns}

\bigskip

$S \dar bCD \dar bCd \dar bcCd \dar bccCd \dar bcccd$

\end{frame}

\sect{LR(1) parsing}

\bee
S \arr aCd \ | \ bCD\\
C \arr cC \ | \ c\\
D \arr d
\eee


\bi
\li $S \dar aCd \dar acCd \dar accCd \dar acccd$
\li $S \dar bCD \dar bCd \dar bcCd \dar bccCd \dar bcccd$
\li At any point, the derivation string must look like one of these:

\bigskip

$aCd$\hfill $ac^+Cd$\hfill $ac^+d$\hfill $bCD$\hfill $bCd$\hfill
$bc^+Cd$\hfill $bc^+d$

\bigskip

\li Whenever we see one of these, we have to know which rule to apply
at what point in the shifting of the string.
\ei

\end{frame}

\sect{LR(1) parsing}
\small
\begin{columns}
\column{0.4\textwidth}

\bee
S \arr aCd \ | \ bCD\\
C \arr cC \ | \ c\\
D \arr d
\eee

\vspace{1cm}

\begin{tabular}{|lr|l|}\hline
Stack & Input & Rule \\\hline
\$ & acccd\$ & shift\\
\$a & cccd\$ & shift\\
\$ac & ccd\$ & shift\\
\$acc & cd\$ & shift\\
\$accc & d\$ & C\ar c\\
\$accC & d\$ & C\ar cC\\
\$acC & d\$ & C\ar cC\\
\$aC & d\$ & shift\\
\$aCd & \$ & S \ar aCd\\
\$S & \$ &    accept\\\hline
\end{tabular}



\column{0.6\textwidth}

\lrtable

\begin{tabular}{|lr|l|}\hline
Stack & Input & Rule \\\hline
\$ & bcccd\$ & shift\\
\$b & cccd\$ & shift\\
\$bc & ccd\$ & shift\\
\$bcc & cd\$ & shift\\
\$bccc & d\$ & C \ar c\\
\$bccC & d\$ & C \ar cC\\
\$bcC & d\$ & C \ar cC\\
\$bC & d\$ & shift\\
\$bCd & \$ & D\ar d\\
\$bCD & \$ & S\ar bCD\\
\$S & \$ &    accept\\\hline
\end{tabular}

\end{columns}

\end{frame}

\sect{DFA for LR parsing}
\small
\begin{columns}
\column{0.4\textwidth}

\bee
S \arr aCd \ | \ bCD\\
C \arr cC \ | \ c\\
D \arr d
\eee

\column{0.6\textwidth}

\lrtable

\end{columns}
\myfigt{lrdfa}

\end{frame}

\sect{Using the DFA in LR parsing}
\small
\begin{columns}
\column{0.5\textwidth}

\lrtable

\begin{tabular}{|lr|l|}\hline
Stack & Input & Rule \\\hline
0 & acccd\$ & shift\\
0 a 1 & cccd\$ & shift\\
0 a 1 c 4 & ccd\$ & shift\\
0 a 1 c 4 c 4 & cd\$ & shift\\
0 a 1 c 4 c 4 c 4 & d\$ & C\ar c\\
0 a 1 c 4 c 4 C 5 & d\$ & C\ar cC\\
0 a 1 c 4 C 5 & d\$ & C\ar cC\\
0 a 1 C 2 & d\$ & shift\\
0 a 1 C 2 d 3 & \$ & S \ar aCd\\
0 S 12 & \$ &    accept\\\hline
\end{tabular}

\column{0.5\textwidth}
\myfigt{lrdfa}
\end{columns}

\end{frame}

\sect{Using the DFA in LR parsing}
\small
\begin{columns}
\column{0.6\textwidth}

\lrtable


\begin{tabular}{|lr|l|}\hline
Stack & Input & Rule \\\hline
0 & bcccd\$ & shift\\
0 b 6 & cccd\$ & shift\\
0 b 6 c 10 & ccd\$ & shift\\
0 b 6 c 10 c 10 & cd\$ & shift\\
0 b 6 c 10 c 10 c 10 & d\$ & C \ar c\\
0 b 6 c 10 c 10 C 11 & d\$ & C \ar cC\\
0 b 6 c 10 C 11 & d\$ & C \ar cC\\
0 b 6 C 7 & d\$ & shift\\
0 b 6 C 7 d 9 & \$ & D\ar d\\
0 b 6 C 7 D 8 & \$ & S\ar bCD\\
0 S 12 & \$ &    accept\\\hline
\end{tabular}

\column{0.5\textwidth}
\myfigt{lrdfa}
\end{columns}

\end{frame}

\sect{More examples in notes on repo.}
\end{frame}

\sect{LR(k) languages, Knuth's theorem}

\begin{theo}
\bee LR(k) \mbox{~languages~} &=& LR(1) \mbox{~languages~}\\
 &=& \mbox{deterministic context free languages}
\eee
\end{theo}
\end{frame}

\sect{LR parsing exercises}

Redo all the solved examples.  Also,
find DFAs and tables for the following languages, and trace some
parses:
\bi
\li $S \ar a \mid b \mid c$
\li $S \ar aSa \mid b$
\li $S \ar ABC $\\
$A\ar a$\\$B\ar b$\\$C\ar c$
\ei
\end{frame}

\sect{Lex and Yacc Style Parsers}

\begin{itemize}
  \item
    \url{http://epaperpress.com/lexandyacc/}
  \item
    \url{https://docs.racket-lang.org/parser-tools/index.html}
\end{itemize}
\end{frame}
    
\end{document}
