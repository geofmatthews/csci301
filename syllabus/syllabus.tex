\documentclass{article}
\usepackage[margin=1in]{geometry}
\usepackage{enumitem}
\usepackage{hyperref}
\usepackage{fancyvrb}
\usepackage{multicol}
\usepackage{color}

\setlength{\parindent}{0pt}

\begin{document}
\centerline{\Large\bf CSCI 301: Formal Languages and Functional Programming}
\centerline{\large\bf Syllabus, Fall, 2018}

\paragraph{Instructor:}
Geoffrey Matthews,
{\tt geoffrey.matthews@wwu.edu}, CF 469, MWF 9:00-9:50

\paragraph{TA:} Evan Ricks, {\tt rickse2@wwu.edu}, CF 163, R 2:00-3:00

\paragraph{Meeting times:} 

  \begin{tabular}{lll|llll}
    \multicolumn{3}{c|}{Lecture}
    &\multicolumn{4}{|c}{Lab}
    \\\hline
 8:00-8:50& MTWF & CF225 &  CRN 41773  &M& CF164 & 12:00-1:50
    \\
& &  & CRN 41774&  W & CF164 & 12:00-1:50 
    \\\hline
    \end{tabular}
	
  You may attend a lab session you did not sign up for on a space
  available basis (you may not bump someone who has registered for
  that section).  There are no labs the first week of class.

\paragraph{Learning environment:}
  I am committed to establishing and maintaining a classroom climate that is
  inclusive and respectful for all students.  Learning includes being able to
  voice a variety of perspectives, and classroom discussion is encouraged. While
  expressed ideas may vary or be opposed to one another, it is important for all
  of us to listen and engage respectfully with each other.


\paragraph{Catalog copy:} Introduction to discrete structures important to
  computer science, including sets, trees, functions, and
  relations. Proof techniques. Introduction to the formal language
  classes and their machines, including regular languages and finite
  automata, context free languages and pushdown automata. Turing
  machines and computability will be introduced. Programming using a
  functional language is required in the implementation of
  concepts. Includes lab.

\paragraph{Goals:}  On completion of this course, students will demonstrate:
\begin{enumerate}
\item
  Thorough understanding of the mathematical definitions of concepts
  important to computer science, including sets, tuples, lists, strings,
  languages, graphs, trees, functions and relations
\item
  The ability to prove basic theorems involving these
  mathematical concepts
\item
  The ability to employ effectively the functional programming style in a
  functional programming language
\item
  Solid understanding of fundamental classes of languages, including
  regular and context free, and their corresponding machines
\item
  Basic understanding of Turing machines and computability
\item
  Basic understanding of important algorithms, including conversion of
  finite automata to different forms, conversion of grammars to machines
\item
  Basic understanding of LL(k) and LR(K) grammars and the parsing
  techniques used for those grammars
\end{enumerate}

\paragraph{Websites:}
\begin{itemize}
\item For class materials:
  \url{https://github.com/geofmatthews/csci301}
\item For turning in homework and
  grading: \url{https://wwu.instructure.com/} 
\end{itemize}


\paragraph{Goals:} This class is an introduction to computer science {\em
  theory}.  This is exactly parallel to the distinction between
theoretical physics and applied physics.  We will use simplified,
abstract, ideal mathematical models of computers, so that we can study
the theoretical limits of what they can accomplish.  We will begin
studying the basic mathematics we need, and then study mathematical
models of computers of increasing complexity and power.  Two classes
of computers, finite state automata and push-down automata, also form
the basis of powerful programming paradigms useful in a variety of
situations.

We will also study the functional programming language {\em Scheme}.
Its simplicity, power, and mathematical elegance will inform our
study of computers in the abstract, and also teach us new styles of
programming.


\paragraph{Math texts:} \mbox{}

Required:
\begin{itemize}
\item \url{http://www.people.vcu.edu/~rhammack/BookOfProof/}
\item \url{http://cg.scs.carleton.ca/~michiel/TheoryOfComputation/}
\end{itemize}
Optional:
\begin{itemize}
\item {\tiny \url{http://ocw.mit.edu/courses/electrical-engineering-and-computer-science/6-045j-automata-computability-and-complexity-spring-2011/lecture-notes/}}
\item
  \url{https://www.tutorialspoint.com/automata_theory/}
\item
  \url{https://www.geeksforgeeks.org/theory-of-computation-automata-tutorials/}
\item
  \url{http://users.utu.fi/jkari/automata/}
\item
  \url{http://maths.mq.edu.au/~chris/notes/second_langmach.html}
\end{itemize} 
\paragraph{Programming texts:} \mbox{}

Required:
\begin{itemize}
\item
      \url{http://ds26gte.github.io/tyscheme/} 
    \item
      \url{http://www.scheme.com/tspl3/}  Note: Use the 3rd edition
      and do {\em not} use the
      4th edition of this book.  Our software is based on R5RS scheme and
      the 3rd edition covers this.  The 4th edition is based on R6RS scheme.
    \item
      The {\em help desk} within Racket leads to documentation on
      the implementation.
\end{itemize}
Optional:
\begin{itemize}
\item
  \url{http://mitpress.mit.edu/sicp/}
\item
  \url{http://www.shido.info/lisp/idx_scm_e.html}
\item
  \url{https://classes.soe.ucsc.edu/cmps112/Spring03/languages/scheme/SchemeTutorialA.html}
\end{itemize}

\paragraph{Software:} {\em DrRacket}, available here: 
\url{http://racket-lang.org/}

\paragraph{Quizzes:}  
Pop quizzes may be handed out at any time.  After working
on them individually we will solve them together in class.
They must be turned in during class for credit.  No makeups.

\paragraph{Exams:} A midterm and a final, according to the schedule below.
The final will be cumulative.

Exams are closed book, with the exception that you may
consult two pieces of paper during the exam.  You may write or print
whatever you wish on both sides.

\paragraph{Laboratory exercises:}
Laboratory exercises will be handed out every week except the first
and last weeks of class and Thanksgiving week (no classes Wednesdays).
They will involve programming in {\em Scheme}.  Each lab is due by
midnight Monday of the following week.  Note that the last lab will be
due Monday of dead week.  No late work will be accepted.

\paragraph{Homework problems:}
Math assignments must be formatted using the \LaTeX\ technical
typesetting system and submitted online.  Turn in both the {\tt .tex}
file and the {\tt .pdf} file.  Do not zip or combine them into one
file.   No late work will be accepted.

Points will be awarded to each assignment based on its difficulty.
Note that there may be an assignment due during dead week.

Ample time to complete the assignments will be allowed.  Late work
will not be accepted.

Pedagogically, it is important that you do as much independent work as
possible.  The assigned homework represents a minimum, but there are
many exercises in the books and online to get more practice at the
math involved.

We will attempt to grade the homework in a timely fashion; this may
necessitate grading only a sample of the problems submitted.

\paragraph{Assessment and grades:}

Your grade will be based on the math homework, labs, quizzes, and the
two exams. Weights are as follows.

Grades will be assigned based on scores as shown.  At the discretion
of the instructor, scores may be scaled.  Awarding $\pm$ is also at
the discretion of the instructor.

\begin{tabular}{|c|c|c|c|c|c|c|}\hline
Homework & Labs & Quizzes & Midterm & Final\\\hline
25\% & 25\% & 5\% & 15\% & 30\% \\\hline
\end{tabular}\hfill
\begin{tabular}{|c|c|c|c|c|c|}\hline
\% & 90-100 & 80-89 & 70-79 & 60-69 & 0-60\\\hline
Grade & A & B & C & D & F\\\hline
\end{tabular}


  
\paragraph{Schedule:}

  \begin{center}
\begin{tabular}{lr|cccc|l|l}                      
& & \multicolumn{4}{c|}{\bf Readings} &  \\
  \multicolumn{2}{c|}{\bf Week} &\em TYS & \em TSPL &\em BoP&\em ITC
& {\bf Labs}  & {\bf Notes} \\\hline  
Sep & 24  & 1 & 1 & I   & &  \\
Oct & 1 & 2,3,4 & 2 & II \& III &   &Lab 1 &\\
Oct & 8 & 5,6,7 & 3 & IV  & 1 & Lab 2&\\
Oct & 15 &  &   &    & 2 & Lab 3&\\
Oct & 22 &  &   &     & 2 & Lab 4&\\
Oct & 29 &  &   &     & 3 & Lab 5&\\
Nov & 5 &  &   &     & 3 & Lab 6 &Midterm Friday, Nov 9\\
Nov & 12 &  &   &     & 4 & Lab 7&\\
Nov & 19 &  &   &     & 5 & Wed Lab 8& Holiday Monday\\
Nov & 26 &  &  &      &  &Mon LaB 8& Holiday Wed-Fri \\                 
Dec & 3   &  \multicolumn{4}{c|}{Review}  &Lab 8 due Dec 3&\\ \hline
Dec & 10  &  \multicolumn{6}{|c}{Final Exam, Mon, Dec 10, 1:00-3:00pm} \\
\end{tabular}

\begin{tabular}{ll}
Book & URL\\\hline
{\em TYS} &       \url{http://ds26gte.github.io/tyscheme/} \\
{\em TSPL} & \url{http://www.scheme.com/tspl3/} \\
{\em BoP} & \url{http://www.people.vcu.edu/~rhammack/BookOfProof/}\\
{\em ITC} & \url{http://cg.scs.carleton.ca/~michiel/TheoryOfComputation/}\\
\end{tabular}

\end{center}

\paragraph{Attendance policy:} Attendance is not required but strongly
  recommended.  Studies show that regular attendance is highly
  correlated with performance.

  You must be in attendance to receive credit for a pop quiz.  There
  are no makeups for pop quizzes.

  You are responsible for all material covered in the lectures, books,
  handouts, or other assigned reading.  If you miss a lecture,
  make sure you get notes from another student.

  If you have an emergency that necessitates your absence from class,
  notify me as soon as possible, preferable before the absence.  I
  will handle each case individually and may, for example, extend the
  due date for the assignment, schedule a make-up exam, or simply
  adjust your remaining scores to determine your grade.

\paragraph{Academic dishonesty:} Please read Appendix D of WWU's Catalog on
  Academic Dishonesty.  It is available online at
  \url{http://catalog.wwu.edu}.

  Unless specified otherwise, all work for this course is meant to
  be done {\bf individually.}  The work that you turn in for a grade
  must be completely your own, or you will be guilty of academic
  dishonesty and could receive an F for the course.

  However, it can be a valiable learning experience to discuss
  work with your fellow students, and this is encouraged.
  However, after working with a colleague, {\bf you may not keep any
    paper or electronic copies of anything you produced together!}
  You may only keep your memories.  In particular, this means that
  {\bf you may not ask for or give help while sitting in front of a
    computer where the assignment is open!}  Also, {\bf you may not
    use anything a colleague has emailed to you!}  Delete the email
  and do not save a copy.

  To help understand what I mean, remember the \fbox{\bf Long Term
    Memory Rule}.  You may discuss, sketch, write things down, use
  your computers, whatever, but after you are done working with your
  fellow students all files must be deleted, whiteboards erased, and
  all papers you created must be destroyed.  You should then watch a
  rerun of {\em the Simpson's}, play a game of ping-pong, take a walk,
  or something else for half an hour. After this you can go back to
  your assignment (alone) and use the knowledge you have now gained.

  It is very easy for experienced software developers like your
  instructor and your TA to detect copied assignments.  Please do not
  put us in a situation where we have to fail you for plagiarism.



\end{document}
