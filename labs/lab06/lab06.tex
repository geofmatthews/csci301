\documentclass{article}
\pagestyle{empty}
\usepackage[margin=1in]{geometry}
\usepackage{fancyvrb,amsmath}
\usepackage{multicol}
\usepackage{xcolor}

\newcommand{\set}[1]{\ensuremath{\{#1\}}}
\newcommand{\power}[1]{\ensuremath{\mathcal{P}(#1)}}

\title{CSCI 301, Lab \# 6}
\author{Fall 2018}
\date{}
\begin{document}

\maketitle
%\setlength{\columnsep}{2em}
%\begin{multicols}{2}


\paragraph{Goal:} This is the fourth in a series of labs that will
build an interpreter for Scheme.  In this lab we will add the {\tt lambda}
special form.

\paragraph{Due:} Your program, named {\tt lab06.rkt}, must be submitted to
Canvas before midnight, Monday, November 12.

\paragraph{Unit tests:}
At a minimum, your program must pass the unit tests found in the
file {\tt lab06-test.rkt}.  Place this file in the same folder
as your program, and run it;  there should be no output.  Include
your unit tests in your submission.

\paragraph{Lambda creates a closure:}

If you consider a simple application of a {\tt lambda} form, it's a lot like
{\tt let}, that you did last week:
\begin{Verbatim}
((lambda (x y) (+ x y)) (+ 2 2) (+ 3 3))    <=>    (let ((x (+ 2 2)) (y (+ 3 3))) (+ x y))
\end{Verbatim}
In each case, a list of arguments, {\tt ((+ 2 2) (+ 3 3))} is evaluated
giving the list of values {\tt (4 6)}, then
the list of variables, {\tt (x y)}, is bound to the list
of values {\tt (4 6)}, and this is used to extend the environment,
and then the expression, {\tt (+ x y)}, is evaluated in the extended
environment, resulting in a value of 10 for the whole expression.

If this were the only use of {\tt lambda}, then it would just be a different
syntax for {\tt let}.

However, the {\tt lambda} is actually more powerful.  A {\tt lambda}
can be created in one environment, and applied in another.  Consider
the following code:
\begin{Verbatim}[frame=single]
  (let ((a 10)
        (f (let ((a 20)) (lambda (x) (+ a x)))))
    (f 30))
\end{Verbatim}
The body of the function {\tt f} is {\tt (+ a x)}.  The function was
called with argument 30, so {\tt x} should be bound to 30.  But what
is {\tt a}? 
Is the answer $10+30$ or $20+30$? 

You can check it out with Racket,
but hopefully you can see that it {\em should} be $20 + 30$.  When the
{\tt lambda} form was {\em evaluated}, {\tt a} was bound to 20.  When
the resulting function, bound to {\tt f}, was {\em applied}, {\tt a}
was 10.

The solution to this is that a {\tt lambda} form creates a special
data structure called a {\em  closure}.  The closure is what is bound
to {\tt f}, in the above example.
A closure consists of three things:
\begin{enumerate}
\item A list of symbols (the variables to be bound when it is called).
\item An expression to be evaluated.
\item A stored environment, the environment in which the closure was
  {\em created}.
\end{enumerate}
The first two are easy to understand.  For example, in the {\tt lambda}
form: {\tt (lambda (x y) (+ x y))}, the list of symbols is {\tt (x y)}
and the expression is {\tt (+ x y)}.  The third thing is 
the environment that was in effect when the closure was created.  In
other words, when the {\tt lambda} form was {\em evaluated}.  A
closure, when created,
stores this third thing, the current environment,
along with the list of variables
and the body.

Because this is lisp, we will represent closures simply
as lists containing the
three items.  Since this is going to be a data type,
let's go ahead and define all the functions that will use
closures, a creator, a predicate, and three accessors:
\begin{Verbatim}[frame=single,label=Closure Data Type]
(define closure
  (lambda (vars body env)  (list 'closure vars body env)))
(define closure?
  (lambda (clos)
    (and (pair? clos)  (eq? (car clos) 'closure))))
(define closure-vars cadr)
(define closure-body caddr)
(define closure-env cadddr)  
\end{Verbatim}
Make sure in your code you {\em only} use these procedures to
handle closures.  Respect the interface!
(If you're interested, you can look up {\bf Racket} {\em structures},
which could have been used instead of lists.)
Note that since closures are just lists, we don't need to write any
special printing procedures to look at them.  They print nicely as
lists already.

So, {\tt lambda} is the new special form for this assignment.
Evaluating a lambda form simply creates a closure out of the
arguments, the body, and the current environment.
For example, if the environment {\tt e1}
is created like this:
\begin{Verbatim}[commandchars=\\\{\}]
(define e1{\color{blue} '((x 5) (y 8) (z 10))} )
\end{Verbatim}
then the following closures would look like this:
\begin{Verbatim}[frame=single,commandchars=\\\{\}]
(evaluate  '(lambda (x) (+ x y)) e1)
      =>  (closure (x) (+ x y) {\color{blue}((x 5) (y 8) (z 10))})
(evaluate  '(lambda (a b c) (cons a (list b c))) e1)
      =>  (closure (a b c) (cons a (list b c)) {\color{blue}((x 5) (y 8) (z 10)})
(evaluate '(let ((x 10)) (lambda (foo) (+ foo foo)) e1)
      =>  (closure (foo) (+ foo foo) ((x 10) {\color{blue}(x 5) (y 8) (z 10))})
\end{Verbatim}
Note that the closure remembers the environment in which it was
created, even if that environment was a special environment, for
example, like the one created by a {\tt let} form.

\paragraph{Applying lambda to some arguments:}

Now that we know how to evaluate a {\tt lambda} form, we need to know
how to {\em apply} a lambda form to some arguments.  For example,
consider the form
\begin{Verbatim}[frame=single]
  ((lambda (x y) (+ x y)) 10 20)
\end{Verbatim}
This is {\em not} a {\tt lambda} form.  It is not even a special form!
It is the {\em application} of
the {\tt lambda} form {\tt (lambda (x y) (+ x y))} to the arguments,
{\tt (10 20)}.  We will evaluate this just like any other
function application!

If we follow our rules for evaluating things that are
not special forms, we would evaluate each of the items in the list
(the {\tt lambda} form, the 10 and the 20), getting this list
(printing the closure as a list):\\
\begin{tabular}{|rcccl|}\hline
  {\tt (}&{\tt (lambda (x y) (+ x y))} & 10 & 20 & {\tt )}\\
         &  \ensuremath{\Downarrow} & \ensuremath{\Downarrow} & \ensuremath{\Downarrow} &\\
  {\tt (} & {\tt (closure (x y) (+ x y) {\color{blue}((x 5) (y 8) (z
      10) ...)}) } & 10 &  20& {\tt )}\\\hline
\end{tabular}\\
And now we have to apply the closure to the list of arguments.  Up to
now, we've only been applying Racket built-in functions to their
arguments, using the Racket {\tt apply} function.  However, Racket
doesn't understand our closures.  We have to figure out how to apply
them ourselves.

So, we will add a new {\tt apply-function} procedure to our
interpreter.  When evaluating a normal (non-special-form) list, we
evaluate each item in the list, and then call {\tt apply-function}
with just two arguments: the {\tt car} and the {\tt cdr} of our
evaluated list of items, the function and its arguments.

{\tt apply-function} will look at its first argument.  If it is a {\tt
  procedure?}, then it calls the Racket built-in {\tt apply}
function.  If it is a {\tt closure?}, it calls {\tt apply-closure}.
Otherwise it should throw an error, reporting an unknown function
type.

{\tt apply-closure} takes two arguments, a closure and a list of
values.  The closure has three components: the variables, the
body, and a saved environment.  This procedure extends the {\em
  saved} environment by appending the variables and their values to
the front (just like {\tt let} did), and then evaluates the body
of the closure in this new, extended environment.

Note that {\tt let} is not the only form that introduces a local
environment, now {\tt lambda} does, too.  Consider the following
code:
\begin{Verbatim}[frame=single]
(let ((f (lambda (a) (lambda (b) (+ a b)))))
  (let ((g (f 10))
        (h (f 20)))
    (list (g 100) (h 100))))  
\end{Verbatim}
Here, the result of applying {\tt f} to 10 creates a closure in an
environment in which {\tt a} is bound to 10.  When we apply {\tt f} to
20, we create a closure in an environment in which {\tt a} is bound to
20.  What should be the result of the call?  Does your interpreter get
that result?

Note also that we can't define recursive functions directly using {\tt
  let}.  What would happen if we tried this?
\begin{Verbatim}[frame=single]
  (let ((f
           (lambda (n)
              (if (= n 0)
                  1
                  (* n (f (- n 1)))))))
      (f 5))
\end{Verbatim}
Try this in the {\bf Racket} interpreter, and also in your
interpreter.
In our next lab, we will define the {\tt letrec} special
form to remedy this.

However, we don't really need to wait for that to define
recursive functions.  We just have to be a little more clever.
For example, what would the following do in {\bf Racket}?  What does
it do in your interpreter?
\begin{Verbatim}[frame=single]
(let ((f
       (lambda (f n)
         (if (= n 0)
             1
             (* n (f f (- n 1)))))))
  (f f 5))
\end{Verbatim}
\end{document}

\end
