\documentclass{article}
\pagestyle{empty}
\usepackage[margin=1in]{geometry}
\usepackage{fancyvrb,amsmath}
\usepackage{multicol}

\newcommand{\set}[1]{\ensuremath{\{#1\}}}
\newcommand{\power}[1]{\ensuremath{\mathcal{P}(#1)}}

\title{CSCI 301, Lab \# 4}
\author{Fall 2018}
\date{}
\begin{document}

\maketitle
%\setlength{\columnsep}{2em}
%\begin{multicols}{2}


\paragraph{Goal:} This is the second in a series of labs that will
build an interpreter for Scheme.  In this lab we process some special forms. 

\paragraph{Due:} Your program, {\tt lab04.rkt}, must be submitted to
Canvas before midnight, Monday, October 29.

\paragraph{Unit tests:}
At a minimum, your program must pass the unit tests found in the
file {\tt lab04-test.rkt}.  Place this file in the same folder
as your program, and run it;  there should be no output.  Include
your unit tests in your submission.

\paragraph{Special forms:}
Special forms are signalled by certain keywords.  Ultimately we
will process the special forms {\tt if}, {\tt cond}, {\tt let},
{\tt lambda}, and {\tt letrec}.  For this lab, however, we will only
do the two conditionals, {\tt if} and {\tt cond}.

You will add another case to your {\tt evaluate} function that will
check for special forms. So your {\tt evaluate} function, adding in
what you did last week, will be able to handle numbers, symbols,
special forms, and lists that are not special forms.  (It will help if
you check for them in that order.)

For this week, a special form is a list
where the {\tt car} of the list is one of the two symbols, {\tt if} or
{\tt cond}.  
Write a procedure to determine if an expression is a special form.
Examples:
\begin{Verbatim}[frame=single]
> (special-form? '(if 1 2 3))
#t
> (special-form? '(cond 1 2 3))
#t
> (special-form? '(+ 1 2 3))
#f
> (special-form? 99)
#f
> (special-form? 'if)
#f
\end{Verbatim}
Note that {\tt special-form?} only checks to see if its argument is a
list, and if the {\tt car} of the list is {\tt if} or {\tt cond}.  It
doesn't check any more of the syntax of the form (but it could).

If your evaluation function finds a special form, it hands the whole
form off to the {\tt evaluate-special-form} function, which acts like
this:
\begin{Verbatim}[frame=single]
> (evaluate-special-form '(if (= 1 2) 3 4) e2)
4  
\end{Verbatim}
The {\tt evaluate-special-form} function checks to see if the {\tt
  car} of the list is either {\tt if} or {\tt cond} (otherwise it
should raise an error), and then does the appropriate thing.

\paragraph{If:}
The special form {\tt if} is clearly not a function, because functions
evaluate {\em all} their arguments.  So, for instance, this:
\begin{Verbatim}[frame=single]
  (if (= 1 1) (print "Hello") (print "Goodbye"))
\end{Verbatim}
would print {\em both} {\tt Hello} {\em and} {\tt Goodbye} if {\tt if}
had been defined as a function.  Clearly, the {\tt if} special form
evaluates only its {\em first argument}, and, on the basis of whether
or not that argument evaluates to \verb|#f| (any value other than
\verb|#f| acts as {\em true}), evaluates {\em either} the second
argument {\em or} the third, not both, accordingly.
The provided environment is used to evaluate the subform.
The one it chooses to evaluate
is the value of the special form.

\paragraph{Cond:}
Once you've figured out how to do {\tt if}, you can tackle {\tt
  cond}.  Consider a typical {\tt cond} special form:
\begin{Verbatim}[frame=single]
  (cond ((= 1 2) 3) ((= 4 5) 6) ((= 7 7) 8) (else 9))
\end{Verbatim}
The {\tt cdr} of this list is clearly a list of alternatives.  You
will write a recursive procedure to go through this list, and evaluate
each of the tests in the items.  For example, the test in the item
\fbox{\tt ((= 4 5) 6)} is \fbox{\tt (= 4 5)}.  As soon as a test
evaluates to \verb|#t|, the other item in the list is evaluated, and
it is returned as the value of the special form.  Again, use the
provided environment to evaluate the selected forms.

But what about {\tt else}?  This is easier than it looks.  {\tt else}
is just a symbol with preset value of \verb|#t|, so put that in the
environment. 
  




\end{document}

\end
