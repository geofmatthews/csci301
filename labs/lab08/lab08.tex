\documentclass{article}
\pagestyle{empty}
\usepackage[margin=1in]{geometry}
\usepackage{fancyvrb,amsmath}
\usepackage{multicol}
\usepackage{xcolor}

\newcommand{\set}[1]{\ensuremath{\{#1\}}}
\newcommand{\power}[1]{\ensuremath{\mathcal{P}(#1)}}

\title{CSCI 301, Lab \# 8}
\author{Fall 2018}
\date{}
\begin{document}

\maketitle
%\setlength{\columnsep}{2em}
%\begin{multicols}{2}


\paragraph{Goal:} This is the last in a series of labs that will
build an interpreter for Scheme.  In this lab we will build our own
parser for Scheme expressions using a simple LL(1) grammar to guide
us. 

\paragraph{Due:} Your program, {\tt lab08.rkt}, must be submitted to
Canvas before midnight, Monday, December 3.
\centerline{\large \bf Submit both {\tt lab07.rkt} and {\tt lab08.rkt} with this
  assignment!} 

\paragraph{Unit tests:}
At a minimum, your program must pass the unit tests found in the
file {\tt lab08-test.rkt}.  Place this file in the same folder
as {\tt lab08.rkt}, and also {\tt lab07.rkt},
and run it;  there should be no output.  Include
your unit tests and both labs in your submission.

\paragraph{From strings to expressions.}

Up until now we let {\bf Racket} handle the conversion from characters
to lists, that is, changing the characters you typed into a file,
for example, \verb|'(+ 1 2)|, into an actual Scheme list
structure with {\tt car}s and {\tt cdr}s.  {\bf Racket} has a builtin
expression reader which does this automatically.

Now we're going to do that part ourselves.  Our unit test file, for
example, can look like this:
\begin{Verbatim}[frame=single]
(require rackunit "lab08.rkt" "lab07.rkt")
(define e2 (list (list '+ +)))
(check equal?
       (parse "(+ 1 2)")
       '(+ 1 2))
(check equal?
       (evaluate (parse "(+ 1 2)") e2)
       3)
\end{Verbatim}
Note that your interpreter will be the one defined in {\tt lab07.rkt},
and will provide {\tt evaluate}, and the parser will be defined
in {\tt lab08.rkt} and provide {\tt parse}.  The parser and the evaluator
should be completely independent.


\paragraph{Input}
We will use the same strategy for input that we used in the {\tt
  rpncalculator.rkt}.  Thus, a string will be converted to a list of
characters, which will be stored in the global variable {\tt
  *the-input*},
and will be accessed {\em only} by the procedure
that takes a string as a parameter, {\tt set-input!},
and the three procedures (which take no
parameters) {\tt end-of-input?}, {\tt current-char},
and {\tt skip-char}.  This will ensure we're doing LL(1) parsing,
since we never look at more than the one character at the front of the
input, we never put characters back on the input to back up and start
over, {\em etc.}

Note: in my RPNcalculator I also used {\tt *the-input*} for meaningful
error messages.  We can allow this, since it is only used in errors
which stop all processing.

\paragraph{LL(1) parsing for Scheme}

The grammar for our Scheme is incredibly simple:
\begin{align*}
  D &\rightarrow 0 \mid 1 \mid 2 \mid 3 \mid ... &\mbox{Digit}\\
  N &\rightarrow DN \mid D &\mbox{Number}\\
  A &\rightarrow a \mid b \mid c \mid d \mid ... &\mbox{Symbolic}\\
  S &\rightarrow AS \mid A &\mbox{Symbol}\\
  E &\rightarrow N \mid S \mid (\ L\ ) &\mbox{Expression}\\
  L &\rightarrow E\ L \mid \epsilon&\mbox{List of expressions}
\end{align*}
Note that, as in our RPNcalculator,
we can use some builtin Scheme procedures
to avoid parts.  For example, {\tt char-numeric?} will identify
digits,
{\tt char-whitespace?} will identify whitespace.

For identifying characters we can use in a symbol (called
``Symbolic'' in the grammar above), we want to include
just about everything except whitespace and the two parentheses.  This
is easy to define:
\begin{Verbatim}[frame=single]
(define char-symbolic?
  (lambda (char)
    (and (not (char-whitespace? char))
         (not (eq? char #\())
         (not (eq? char #\))))))  
\end{Verbatim}

\paragraph{Recursive descent procedures}

We will define four recursive descent procedures to parse Scheme:
\begin{enumerate}

\item{\tt parse-expression} is the top-level procedure, and gets the ball
rolling. From the grammar we know there will be only three kinds of
expressions.  The LL(1) character of the grammar tells us how
to do the processing:
\begin{itemize}
\item If we find a digit, call {\tt parse-number}.
\item If we find a symbolic character, call {\tt parse-symbol}.
\item
  If we find a left parenthesis, remove that character from the
  input and call {\tt parse-list}.
\item
  Anything else should be an error.
\end{itemize}

\item{\tt parse-number} is handled
 just as in the RPNcalculator: keep scanning digits until
you find something not a digit, multiplying by 10 and adding along the
way.  When you get to the end of a number, though, don't push it on a
stack, return it as the value of the expression.  We are not building
an interpreter here (like the RPNcalculator), we're building  a
parser.  A properly parsed number is ... the number itself!

\item{\tt parse-symbol} is handled very similarly to numbers:  keep
scanning as long as the input is symbolic, and concatenate all the
characters into a symbol when you're done.  You may find the builtin
{\bf Racket} routine {\tt string->symbol} helpful.  Return the symbol
as the value of the procedure.

\item{\tt parse-list} is a bit tricky, but
  the grammar tells us we only need to
handle two cases:
  \begin{itemize}
    \item
 If the input is a right
 parenthesis, the return \verb|'()|.  We found the end of the list.
\item
 Otherwise, call {\tt
  parse-expression}, then call {\tt parse-list}, {\tt cons}
the two results together and return that.
\end{itemize}
\end{enumerate}
That wasn't so hard, was it?  Note:  I left out discussing the
handling of whitespace, but that's not hard.  Look at what we did in
the RPNcalculator.


\paragraph{Parse}
The {\tt parse} procedure is a simple interface to the grammar:
\begin{Verbatim}[frame=single]
(define parse
  (lambda (str)
    (set-input! str)
    (parse-expression)))
\end{Verbatim}
It is usually the case that when you have a complex recursive descent
parser, you need one more ``top-level'' routine to get everything
rolling. 

\paragraph{Program files: optional}

Note that our parser only handles strings typed in, but that we
usually want an interpreter to interpret program files, not strings.
The solution is pretty trivial:
check out the {\bf Racket} procedure {\tt file->string}.

With a little work you could get your interpreter to interpret itself...


\end{document}

