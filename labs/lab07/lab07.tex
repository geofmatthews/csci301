\documentclass{article}
\pagestyle{empty}
\usepackage[margin=1in]{geometry}
\usepackage{fancyvrb,amsmath}
\usepackage{multicol}
\usepackage{xcolor}

\newcommand{\set}[1]{\ensuremath{\{#1\}}}
\newcommand{\power}[1]{\ensuremath{\mathcal{P}(#1)}}

\title{CSCI 301, Lab \# 7}
\author{Fall 2018}
\date{}
\begin{document}

\maketitle
%\setlength{\columnsep}{2em}
%\begin{multicols}{2}


\paragraph{Goal:} This is the fifth in a series of labs that will
build an interpreter for Scheme.  In this lab we will add the {\tt letrec}
special form.

\paragraph{Due:} Your program, named {\tt lab07.rkt}, must be submitted to
Canvas before midnight, Monday, November 26 (after Thanksgiving).

\paragraph{Unit tests:}
At a minimum, your program must pass the unit tests found in the
file {\tt lab07-test.rkt}.  Place this file in the same folder
as your program, and run it;  there should be no output.  Include
your unit tests in your submission.

\paragraph{Letrec creates closures that include their own definitions.}

Consider a typical application of {\tt letrec}:
\begin{Verbatim}[frame=single]
  (letrec ((plus (lambda (a b) (if (= a 0) b (+ 1 (plus (- a 1) b)))))
           (even? (lambda (n) (if (= n 0) true (odd? (- n 1)))))
           (odd? (lambda (n) (if (= n 0) false (even? (- n 1))))))
    (even? (plus 4 5)))
\end{Verbatim}
{\tt plus} is a straightforward recursive function.  {\tt even?} and
{\tt odd?} are mutually recursive functions, each one requires the
other.

If we use {\tt let} instead of {\tt letrec}, we will evaluate the {\tt
  lambda} forms in the current environment, and none of the three
functions will be defined in that environment.  Each of the closures
will contain a pointer to an environment in which the recursive
functions are {\em not} defined.  Thus, we cannot simply
use {\tt let}.

We want the closures to close over an environment in which {\tt plus},
{\tt even?} and {\tt odd?} {\em are} defined.  To do this, we will follow
this strategy.
\begin{enumerate}
\item To evaluate a {\tt letrec} special form,
  we first run through the variable-value pairs in the {\tt letrec}
    expresion as if it was a simple {\tt let}.  In other words, we go
    ahead and create the closures with the {\em wrong} environment.
    We will fix this later.
  \item Anything else in a {\tt letrec} is also handled in the {\tt
    let} fashion, for example
\begin{Verbatim}
(let ((a 2))
   (letrec ((x (+ a a)))
      (+ x x)))
\end{Verbatim}
will just return 8.
You can reuse your old {\tt let} code from  previous
labs for this part.
\item In the course of evaluating a {\tt let} expression,
  you created a mini-environment, and appended that to the current
  environment, to get a new environment.  We will need pointers to all
  three of these in what follows.  For this writeup, I'm going to call
  them the {\tt OldEnvironment}, the {\tt MiniEnvironment}, and
  the {\tt NewEnvironment}.  They are illustrated as follows:
\begin{Verbatim}
((x 5) (y 10) ...)                                     <= OldEnvironment
((plus ...) (even? ...) (odd? ...))                    <= MiniEnvironment
((plus ...) (even? ...) (odd? ...) (x 5) (y 10) ...)   <= NewEnvironment
\end{Verbatim}
\item At this point, the closures in {\tt MiniEnvironment}
  contain pointers to {\tt OldEnvironment}.  We need to change these
  to point to {\tt NewEnvironment}.
\item If there are any closures in {\tt OldEnvironment}, however, they
  are already correct, so we don't want to change them!

\item So, we need to loop through just the variable-value pairs in {\tt
  MiniEnvironment}.  If any variables are bound to {\em closures}, we
  change the {\em saved} environment pointer inside the closure to
  point to {\tt NewEnvironment}.

 Make sure you loop through only the variable-value pairs in {\tt
  MiniEnviroment}. Note that {\tt NewEnvironment} includes both {\tt
  MiniEnvironment} and {\tt OldEnvironment}. So we {\em don't } want
  to loop through all the closures in {\tt NewEnvironment}.

\item Since we need to change a data structure that already exists,
  this is definitely {\em not} functional style programming.  In fact,
  lists in {\bf Racket} are immutable!  So we {\em cannot} use lists
  any more to represent closures.

 Look up the documentation in {\bf Racket} on mutable lists.
  You'll find procedures such as {\tt mcar}, {\tt mcdr}, and {\tt
    mcons}, which handle mutable lists just the way {\tt car}, {\tt
    cdr} and {\tt cons} handle immutable lists.

  But you will also find procedures such as {\tt set-mcar!} and
  {\tt set-mcdr!} for changing existing lists into new ones.

  Using mutable lists allows us to change the implementation of
  closures so that we can change the environment inside:

\begin{Verbatim}[frame=single,label=Mutable Closures]
(define closure
  (lambda (vars body env)
    (mcons 'closure (mcons env (mcons vars body)))))
(define closure?
  (lambda (clos) (and (mpair? clos) (eq? (mcar clos) 'closure))))
(define closure-vars
  (lambda (clos) (mcar (mcdr (mcdr clos)))))
(define closure-body
  (lambda (clos) (mcdr (mcdr (mcdr clos)))))
(define closure-env
  (lambda (clos) (mcar (mcdr clos))))
(define set-closure-env!
  (lambda (clos new-env) (set-mcar! (mcdr clos) new-env)))
\end{Verbatim}


If you take the previous lab and replace the old closure
implementation with this one, everything should work as before.  Now
aren't you glad you respected the interface in the last lab?

 If your previous lab doesn't work with this new implementation
  of closures, fix it!

\item After we replace all the closures in {\tt MiniEnvironment} with
  pointers to {\tt NewEnvironment}, we can now evaluate the body of
  the {\tt letrec} form, using {\tt NewEnvironment}.  Yay!  Recursive
  functions! 

\item Check the unit test file for some tricky examples!  

  Can you believe you've written an interpreter for such a complex language? 
\end{enumerate}

\end{document}

